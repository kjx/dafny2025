\section{Conclusion}
%\vspace*{-2mm}

In this short paper we've described our initial experience with
modelling local stores --- first class regions --- with Dafny.
Starting from an historical perspective, we've discussed how Utting's
local store ADT could be modelled in Dafny, albeit not in a way Dafny
programmers could use such an ADT effectively.

Then, drawing on our work-in-progress models of ownership in
programming languages \cite{dafnydala-ftfjp2024}, we have shown how
Dafny's object model can be extended to incorporate local stores into
object implementations, and how we can use those local stores to
support reasoning about aliasing.

On the whole we think this effort has been worthwhile: using
explicit local stores seems no more clumsy than Dafny's native dynamic
framing, and can be more precise and more flexible than the
autocontracts mechanism.  Furthermore, by providing programmers with a
notation to describe nesting relationships between local stores, we
speculate that information about such relationships may be employable
e.g.\ to generate better code for Rust or other langauges where object
lifetimes, nesting, and ownership are important.  On the other hand,
the lack of reflection in plain Dafny means that the larger prize,
tighter pelluicid integration of Dafny objects and ownership, would
require at least compiler plug-ins, or language extensions like Linear
Dafny \cite{linear-dafny-oopsla2022}.

The main disadvantage we found from using Dafny is that programmer
time to achieve verification always seems to take longer than any
estimate --- even though when working with 
Dafny, we often felt that just "one more assertion" would be enough to
verify our entire model. We consider this comes from the automatic
yet opaque style of verification that is intrinsic to Dafny. Dafny verification acts as a intermittent positive reinforcement on a variable ratio schedule, like a slot machine, with the same addictive qualities
\cite{dafny-europlop2024}.

%% sstudents get feedback
%% expectations are clear
%% can work local or remotely (COVID)
%% sthudents always nkow where they are
%% students can choose how much time and effort to invest
%% mastery means can demonstrate a ``taste'' of the topic 


% \vspace*{-1mm}

% \begin{verse}
%   \textit{
% Getting code to work is one thing.\\
% Proving it does what it's supposed to is something else.\\
% Convincing Dafny you've proved it does what it's supposed to\\
% ~~~~is something else entirely.}
% \end{verse}
% \vspace*{-10mm}
% \begin{flushright}
% ``Motto for a Software Correctness Course''\\
% Thomas J. ``Tad'' Peckish (attrib.), twitter, Oct 4 2020
% \end{flushright}
